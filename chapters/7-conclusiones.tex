\chapter{Conclusiones}
\label{cap:conclu}
% Intro general
% Capítulo de Wand multithreading
% Conclusión 
% Capítulo de predicción de tiempos
% Conclusión
% Capítulo de estrategias de planificación
% Conclusión
% Conclusión objetivos específicos
% Conclusión general
% Aportes al área
% Trabajo futuro
Por medio del presente trabajo se ha llevado a cabo un estudio del procesamiento y planificación de transacciones de lectura que llegan a un motor de búsqueda haciendo uso de una máquina multinúcleo, en el que se adaptaron diferentes estrategias de planificación del estado del arte al contexto de un motor de búsqueda, y se evaluaron los rendimientos de cada una de ellas mediante el procesamiento de consultas por lotes. Adicionalmente se propone una estrategia de procesamiento de consultas basada en unidades de trabajo, en el que cada consulta es dividida en unidades de procesamiento y los diferentes hilos de ejecución compiten por procesar cada unidad.

El sistema implementado para resolver una transacción de lectura que llega al sistema es flexible a hacer uso de diferentes números de hilos de ejecución; esta capacidad se logra ya que se implementó dos versiones paralelas del algoritmo Wand, la primera es una versión con \textit{heaps} locales y la otra hace uso de un solo \textit{heap} compartido. Los resultados arrojan que utilizando la muestra de la Web Gov2 y obteniendo el conjunto de los \textit{top-100} mejores documentos, el enfoque con \textit{heap} compartido posee mejor rendimiento. 
% BMW

Con respecto a los predictores de rendimiento de transacciones de lectura, el método ML basado en una regresión lineal múltiple obtiene mejor rendimiento en la predicción que el método RN basado en redes neuronales, esto utilizando el mismo conjunto de 42 descriptores para ambos métodos. Esto no descarta que exista una mejor solución para el método RN que incluso pueda mejorar en rendimiento al método ML, para esto se debe hacer una mejor clasificación de descriptores utilizados para la creación del modelo.
 
Las estrategias de planificación por bloques no poseen un buen rendimiento cuando el objetivo es procesar grandes cantidades de transacciones de lectura por lotes, debido a la pérdida de tiempo que existe entre la sincronización de bloques. Por otro lado, la estrategia de procesamiento de consultas por unidades de trabajo parece ser la mejor forma de reducir el tiempo total de procesar grandes cantidades de consultas por lotes yy al mismo tiempo asegurar una cota superior de tiempo para cada una de ellas.

En condiciones ideales de predicciones de tiempo, la estrategia de unidades de trabajo posee mejor rendimiento que la estrategia 1TQ creada como \textit{baseline}. A medida que el tamaño de los lotes crece, la diferencia de rendimiento es menor, debido a que existen menos sincronizaciones entre lotes, lo que implica una menor pérdida de eficiencia para la estrategia 1TQ.

En el presente contexto de procesamiento de transacciones de lectura en una máquina multicore, probablemente sea una mejor idea enfocar esfuerzos en crear un predictor más simple y preciso que los vistos en el presente trabajo por sobre intentar reordenar las consultas, de esta se obtendrán mejores tiempos para el procesamiento del conjunto completo de consultas.

Finalmente es posible afirmar que se ha cumplido con todos los objetivos planteados al comienzo de este trabajo. Se ha desarrollado dos algoritmos de procesamiento de transacciones de lectura basados en el algoritmo Wand, una con \textit{heap} compartido y otra con \textit{heaps} locales. Se han creado estrategias de planificación \textit{online} que reordenan y adaptan dinámicamente las consultas al tamaño de las estructuras de datos disponibles; estas estrategias fueron adaptadas al contexto de un motor de búsqueda vertical en el que se procesan transacciones de lectura por lotes. Finalmente la evaluación para los métodos de predicción se hizo mediante la medición del RMSE y error relativo porcentual promedio (ERP); por otro lado, la evaluación para los métodos de planificación y procesamiento se hizo en base al tiempo que tarda cada una de ellas en procesar el conjunto completo de consultas.

\section{Trabajo futuro}
\label{conclu:trabajofuturo}
Con respecto al trabajo futuro, sería interesante analizar el comportamiento que tienen las estrategias de procesamiento y de planificación para diferentes tamaños del conjunto \textit{top-K} y el impacto que pueda tener sobre los métodos de aprendizaje.
También es importante trabajar en la creación de un método de predicción con mayor precisión que los presentados en este trabajo, resulta interesante estudiar si es posible reducir el número de variables del modelo multilineal manteniendo o mejorando el error. Adicionalmente hacer un análisis detallado de cada uno de los descriptores utilizados y ver la posibilidad si es posible crear un método de predicción nuevo más simple y preciso.
El modelo creado hasta ahora para procesar transacciones de lectura es flexible a hacer pausas entre los lotes; sería interesante agregar el procesamiento de las transacciones de escritura a este modelo y estudiar el comportamiento y el impacto que tienen estas transacciones sobre el sistema.