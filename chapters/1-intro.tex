\chapter{Introducci\'on}
\label{cap:intro}


% ***************************************************************************************************** MOTIVACIÓN

\section{Antecedentes y motivaci\'on}
\label{intro:motivacion}

% CONCEPTOS PROVENIENTES DE LA WEB 2.0

La Web 2.0 o ``Web Social'' se caracteriza por ser una Web centrada en el usuario, la participación, la interactividad y la colaboración \citep{Murugesan:2007}.\textit{Youtube}\footnote{http://www.youtube.com/}, \textit{Facebook}\footnote{https://www.facebook.com/}, \textit{Flickr}\footnote{http://www.flickr.com/} ahora son compañeros cotidianos y habituales en el mundo tecnológico que nos toca vivir. Antes de este cambio cultural, la Web 1.0 estaba centrada principalmente en el contenido, el cual era actualizado unilateralmente por usuarios expertos; era principalmente comunicación de una sola vía. En cambio, la Web 2.0 elimina las barreras entre productores y consumidores de información, creando canales de doble vía, empoderando al usuario final como creador activo de información. Este cambio produjo una verdadera revolución social del uso de las tecnologías en la red. Sea dicho de paso, esta revolución se vio facilitada por la incorporación de la tecnología de la  ``sindicación'' la cual permite retroalimentación en línea sobre la información de productores y consumidores. En definitiva la Web 2.0 es una forma distinta de usar la Web como canal de comunicación  \citep{Padula:2009} que permite a los usuarios finales, además de publicar, interactuar con los distintos contenidos de la Web (textual, multimedia, etc) de diversas formas.

El masivo uso de las aplicaciones de la Web 2.0 ha poblado Internet de una enorme cantidad de información, por este motivo, y para ayudar a los usuarios en la búsqueda de información, se desarrollan dos tipos de sistemas: los motores de búsqueda y sistemas de recomendación (en adelante SR). Estos últimos entregan información filtrada de acuerdo al comportamiento de otros usuarios (meta-información), de las preferencias propias, ajenas y de los atributos de la información buscada. El comportamiento de los usuarios puede ser modelado como un conjunto de interacciones de valor colaborativo que permiten obtener información sobre la colaboración entre usuarios explícita o implícita.  Estas interacciones pueden ser de distintos tipos \citep{Dijkman:2011}, e incluyen actividades como  etiquetar (\textit{tagging}), comentar (commenting), valorizar (\textit{rating}) entre otros. 

El área de investigación de los SR nace a mediados de los 90’, cuando \textit{GroupLens}\footnote{http://www.grouplens.org/} publica el primer artículo, que se basa en el filtrado colaborativo para generar recomendaciones de noticias \citep{Herlocker:1999}. En el estado del arte \citep{Adomavicius:2005} se plantea que esta área se ha mantenido en expansión debido a que provee un gran número de aplicaciones prácticas que ayudan a los usuarios a encontrar información de utilidad.

En la actualidad existen varios SR, enfocados a distintos dominios de aplicación. Algunos ejemplos son sitios como \textit{MovieLens}\footnote{http://movielens.umn.edu/login} y \textit{Netflix}\footnote{www.netflix.com} para recomendación de peliculas; \textit{Google News}\footnote{news.google.com} para recomendación de noticias; \textit{LastFm}\footnote{www.last.fm} para recomendación de música, y \textit{Amazon}\footnote{www.amazon.com} para recomendación de \textit{e-commerce}, entre otras. En este mismo ámbito y con el objetivo de mejorar su SR, \textit{Netflix} creó un concurso llamado \textit{“Netflix Prize”}\footnote{http://www.netflixprize.com} que premiaba con un millón de dólares a quienes lograran mejorar su algoritmo de recomendación basado en la métrica \textbf{RMSE} (error cuadrático medio). Este concurso motivó la investigación en el área promoviendo nuevas formas de abordar el problema de recomendación basada en \textit{rating}. 

Los SR tradicionales se basan en dos dimensiones: el usuario y el ítem para calcular una función de utilidad \citep{Adomavicius:2005}. Básicamente se construyen perfiles de usuario e ítem, como también se modelan las interacciones de los usuarios hacia los ítems. Sin embargo, en el desarrollo de la investigación en el área se están modelando nuevas dimensiones con el objetivo de mejorar el proceso de recomendación \citep{Adomavicius:2011}. Ejemplos de estas dimensiones son el prestigio y afinidad entre usuarios para medir la confiabilidad de la recomendación \citep{Victor:2011}. La geolocalización \citep{Panagiotis:2011} basándose en la ubicación física y/o virtual del usuario. Otro conjunto de dimensiones importantes se basan en el contexto de interacción del usuario mejorando así el proceso de recomendación \citep{Adomavicius:2011}. Luego  estos sistemas se han sobre-especializado con el objetivo de entregar recomendaciones de mayor relevancia para el usuario final.

En las empresas se hace cada vez más necesario contar con SR para aumentar las probabilidades de selección de sus productos y/o servicios por parte de sus clientes. Cada empresa cuenta con registros de las interacciones de sus clientes en diversos repositorios de datos que suelen ser transversales a los sistemas de la empresa. Además, existe la información proveniente de las aplicaciones basadas en la Web 2.0, que proveen API’s para consultar información de sus clientes, como también aplicaciones innovadoras de las empresas que se basan en la ideología de la Web 2.0.

En el área de investigación de los SR es difícil reproducir y extender los resultados obtenidos en las distintas publicaciones \citep{Ekstrand:2011}. Por lo tanto, se debe re-implementar los SR para un dominio particular donde éstos son usados. En este contexto, han aparecido trabajos que intentan aportar estándares en la representación, construcción y validación de SR que permitan la construcción sistemática de SR. Por ejemplo en \textit{Lenskit} \citep{Ekstrand:2011} se provee un \textit{framework} extensible y mantenible para la construcción y evaluación de SR basados en filtrado colaborativo. Por otro lado en \textit{Synergy} \citep{Babar:2010} se plantea un modelo de datos que representa las distintas interacciones que realizan los usuarios con los ítems. \cite{Palomino:2012} plantea un \textit{framework} que busca organizar y estandarizar el área de los sistemas de recomendación en dos aspectos: un modelo de representación de datos, y una arquitectura para proveer servicios de recomendación en una red social. 

\section{Descripci\'on del problema}
\label{intro:problema}

%El problema de esta tesis es de transferencia tecnológica de los SR. Cada vez existe mayor presión para disponer de prototipos rápidos de SR para propósitos de investigación y de comercialización a partir de los avances actuales del área. Sin embargo, no son comunes las guías que ayuden a ese propósito. Sumado a esto, los SR son fuente de tremendos desafíos prácticos que aún no tienen una respuesta definitiva tales como:

La representación y construcción de SR son fuente de los siguientes desafíos:

% elimine los cambios en las preferencias de los clientes.
\begin{enumerate}
	%\item Cambios en las preferencias de los clientes.
	\item La Web 2.0 proporciona nuevas formas de interacción para los usuarios. Luego, existen dificultades para modelar múltiples interacciones para que estas sean usadas por los SR \citep{Babar:2010}.
	\item Los SR para obtener mejores resultados sobre un dominio específico se acoplan demasiado a la data \citep{Babar:2010}. Esto provoca una sobre-especialización del SR, dificultando la posibilidad de reutilizar la solución en otro dominio de aplicación.
	\item Un SR puede degradar sus resultados en el tiempo debido a cambios en la data referente a la interacción  de los usuarios por lo que debería ser re-entrenado o cambiado por otro.
	\item Las mejoras obtenidas en los algoritmos son difíciles de generalizar para todos los usuarios del sistema,  por ejemplo algunos usuarios pueden usar un campo muy reducido de opciones de interacción y otros más.
\end{enumerate}

Un ingeniero a cargo de realizar un SR espera contar con estándares y patrones para la construcción, por lo tanto no suele resolverlo de manera \textit{ad-hoc} como lo hace un investigador del área de los SR. En efecto, un investigador tiende a especializarse para resolver problemas científicos y no se preocupa de proveer estándares para que su solución sea generalizable y/o reutilizable.

% se elimina lo relacionado con los Ingeniero e Investigador

En definitiva, el problema de esta tesis está relacionada con la generalización, contextualización y transferencia de los SR. Esto puede expresarse en la siguiente pregunta que guía este trabajo de tesis: ¿Es posible contar con un \textit{framework} que permita representar y construir SR de distinta complejidad en diversos dominios de aplicación?


\section{Objetivos y solución propuesta}

A continuación se presenta el objetivo general del proyecto que se concreta con el cumplimiento de los objetivos específicos. Para finalmente declarar los alcances del trabajo de tesis.

\subsection{Objetivo General}

Construir un \textit{framework} que permita la representación y construcción de SR.

\subsection{Objetivos Espec\'ificos}

\begin{enumerate}
	\item Sistematizar el proceso de construcción de SR desde la literatura actual del área.
	\item Caracterizar las dimensiones emergentes para los SR.
	\item Diseñar un modelo orientado a eventos de representación de datos para SR.
	\item Diseñar un conjunto de operaciones para la obtención de datos desde el modelo de datos propuesto.
	\item Diseño e implementación de una herramienta para construir SR.
	\item Construir SR’s basados en el \textit{framework} propuesto.
	\item Publicar los resultados de la investigación en una revista de la especialidad.
\end{enumerate}

\subsection{Alcances}

El trabajo de tesis propuesto tiene los siguientes alcances:
\begin{itemize}
	\item Dada la amplia variedad de SR que se encuentra en la literatura; la solución aunque es genérica se ejemplifica sólo a los SR de filtrado colaborativo \textit{user-user} y de \textit{tagging social} basados en técnicas de \textit{clustering}.
	\item No es concluyente en la mejora de rendimientos, sino solamente en la eficacia del \textit{framework} propuesto.
	\item Sólo se validará con el desarrollo de algoritmos que trabajen exclusivamente con \textit{tags} y \textit{ratings} como dos tipos de eventos colaborativos posibles.
	\item Este producto de software sólo contempla la construcción de un API y una implementación de esta para la construcción de los SR.
	\item Tanto el API y la implementación serán escritas en \textit{Java Standard Edition 7}.
	\item Este producto de software no dispondrá extensiones \textit{Windows} ni \textit{Unix}, \textit{Linux}. Sólo podrá ejecutar en la máquina virtual \textit{Java} disponible en estos sistemas operativos.
	\item No se contempla establecer los procedimientos de evaluación de algoritmos dentro de la herramienta construida.
\end{itemize}

\subsection{Soluci\'on propuesta}
\label{intro:solucionpropuesta}

Este proyecto de investigación será del tipo investigación aplicada \textbf{I+D} (Investigación
+ Desarrollo):
\begin{enumerate}
\item \textbf{Investigación}: se propondrá un modelo que permita la representación de datos basándose en un framework proveniente del área de los sistemas colaborativos y un método de construcción de SR que se encuentren en el estado del arte del área.
\item \textbf{Desarrollo}: se construirá una API basada en el modelo y método propuesto. 
\end{enumerate}

Finalmente se construirán dos SR usando el API anterior para probar la eficacia del modelo propuesto.

\subsection{Caracter\'isticas de la solución}
\label{intro:caracteristicassolucion}

Las características de la solución propuesta son las siguientes:

\begin{itemize}
	\item Una especificación o API\footnote{\textit{Application Programming Interface}} extensible basada en patrones de diseño que aseguren la flexibilidad requerida basada en un bajo acoplamiento, mantenibilidad y alta cohesión.
	\item Se construirá una herramienta basada en el API propuesta que permita la representación de datos basado en un método para construir SR. 
	\item La representación de datos se realizará bajo un esquema orientado a eventos que permita	 modelar el comportamiento de los usuarios en la Web 2.0. Dada esta forma de representación se construirá un conjunto de operaciones para la adquisición de datos del modelo como entrada para los SR.
	\item La construcción de algoritmos se realizará caracterizando el proceso de recomendación.
	\item Prueba empírica con la herramienta de software mediante la construcción de SR, que expresen la utilidad del modelo propuesto.
\end{itemize}

\subsection{Prop\'osito de la solución}
\label{intro:propositosolucion}
Los propósitos de la solución son los siguientes:

\begin{itemize}
\item Proveer un modelo de representación de datos extensible. Para la concreción de lo anterior, se utilizará un \textit{framework} conceptual del área de los sistemas colaborativos denominado \textit{3-Ontology} propuesto por \citep{Leiva:2002} que permite modelar las interacciones dentro de un ambiente colaborativo.
\item Un método que permita la construcción de SR del estado del arte.
\item Proveer al área de los SR una herramienta de uso personal o empresarial extensible basada en el modelo de representación de datos y el método de construcción.
\end{itemize}


\section{Metodolog\'ia y herramientas de desarrollo}

\subsection{Metodolog\'ia}

La metodología usada para este trabajo como se explicó anteriormente se basa en un proyecto de investigación aplicada I+D.

En el ámbito de la investigación se realizará un estudio exploratorio sobre la representación de datos en los SR y su relación con un modelo particular de representación del contexto de trabajo en los sistemas colaborativos. Para lograr este objetivo se modelará un \textit{framework} para SR. Por lo tanto, el tipo de investigación que se llevará a cabo será de índole exploratorio donde el objetivo es examinar  un tema o problema de investigación poco estudiado o que no ha sido abordado antes \citep{Sampieri:2005}. Se considera exploratorio ya que se ha encontrado poca literatura que aborde la representación de datos en SR. A partir de esto se contemplan las siguientes etapas:

\begin{itemize}
\item Revisión del estado del arte de SR para realizar una categorización.
\item Revisión del \textit{framework} conceptual \textit{3-Ontology} para especificar los conceptos básicos del área que serán usados en la solución.
\item Modelamiento del proceso de generación de SR.
\item Modelamiento de dimensiones emergentes de SR.
\item Comparación con otros \textit{framework}.
\item Evidenciar la completitud del \textit{framework} mediante la representación teórica de diversos tipos de SR.
\item Dar evidencia empírica del \textit{framework} mediante la construcción de SR de filtrado colaborativo.
\end{itemize}

La metodología de desarrollo del software usada se basa en la filosofía ágil de \textbf{SCRUM}\footnote{https://www.scrum.org/}. Sin embargo dado que el trabajo fue realizado de forma personal, no se especificaron los roles de la metodología. Se justifica una metodología ágil ya que los requerimientos de la herramienta usada varían durante el desarrollo \textit{framework}. Luego, se tiene la flexibilidad necesaria para abordar cambios de requerimientos a lo largo del desarrollo. La documentación que se genera es simple y básica:
\begin{itemize}
	\item Documentación del código mediante \textit{Javadocs}\footnote{http://www.oracle.com/technetwork/java/javase/documentation/index-jsp-135444.html}.
	\item Diseño de clases.
	\item Modelo de componentes: especifica las interacciones de los distintos componentes.
	\item Diagrama de módulos.
\end{itemize}

\subsection{Herramientas de desarrollo}

Las herramientas que fueron usadas para la realización de este trabajo de tesis son las siguientes:

\begin{itemize}
	\item Sistema operativo: \textit{Linux} con su distribución \textit{Ubuntu 12.04}.
	\item \textit{Java SE Development kit 7}\footnote{http://www.oracle.com/technetwork/java/javase/overview/index.htmltwork/java/javase/overview/index.html}: se utiliza para el desarrollo del software. Se justifica su uso por ser un lenguaje orientado a objetos de alto nivel que permite una representación legible del \textit{framework} construido. Además, posee una comunidad bastante amplia y gran número de librerías que facilitan el desarrollo.
	\item \textit{TeXstudio}\footnote{texstudio.sourceforge.net}: herramienta para la construcción de documentos en \textbf{LaTeX}.
	\item \textit{Eclipse Kepler}\footnote{http://www.eclipse.org/kepler/}: el \textbf{IDE} \textit{eclipse} en su versión \textit{kepler} que es usada para desarrollo de aplicaciones estándar de \textit{Java}.
	\item \textit{DIA}\footnote{http://dia-installer.de/}: para el desarrollo de los diagramas subyacentes al código y los modelos construidos.
	\item \textit{Notebook} personal:\textit{ Toshiba L505D}. \textit{AMD Turion(tm) II Dual-Core Mobile M520 × 2}. 4 GB de memoria RAM.
	\item \textit{Gradle}\footnote{http://www.gradle.org/}: herramienta que permite el control del ciclo de vida del software, desde su construcción al despliegue. Adopta los estándares propuestos por \textit{Apache Maven} como estructura de directorios y manejo de dependencias.
	\item \textit{Git}\footnote{http://git-scm.com/}: herramienta para el control de versiones del código fuente y documentos asociados al trabajo de tesis.
	\item \textit{BitBucket}\footnote{http://bitbucket.org/}: repositorio remoto de \textit{Git} donde se administra el control de versiones.
	\item Servidor \textit{Huelen} del Departamento de Ingeniería Informática, \textbf{USACH}.
\end{itemize}

\section{Resultados obtenidos}

La representación de SR basado en el \textit{framework} conceptual \textit{3-Ontology} provee una forma de situar las interacciones de los usuarios hacia los ítems dentro de aplicaciones colaborativas como son los SR. Por lo tanto, se asume que las interacciones tienen un contexto que les da un sentido espacial, temporal y social. La herramienta construida RBOX 2.0 se basa en el modelo propuesto y cumple con las propiedades sistemáticas de mantenibilidad, flexibilidad, reusabilidad y escalabilidad. La eficacia del modelo se valida mediante la construcción de un SR de filtrado colaborativo \textit{user-user} para \textit{Movielens}, y otro SR de \textit{Tagging Social} bajo un \textit{dataset} propietario de una empresa de noticias. En cada SR construido se muestra la correspondencia del dominio de aplicación al modelo propuesto.

\section{Organizaci\'on del documento}

En el Capítulo 2 se presenta un marco teórico donde se exponen los fundamentos del trabajo como son los conceptos básicos de los SR, características emergentes de los SR dentro del contexto de la información social y el \textit{framework} conceptual \textit{3-Ontology}. A continuación, en el Capítulo 3 se realiza la definición del modelo propuesto comenzando con una revisión de los trabajos relacionados para luego definir formalmente el modelo. En el Capítulo 4 se presenta el diseño e implementación de RBOX 2.0, la herramienta de software basada en el modelo de representación propuesto. Luego en el Capítulo 5 se presentan dos casos de estudio donde se valida la eficacia del modelo propuesto. En el Capítulo 6 se presentan las conclusiones, para finalmente terminar con las referencias del trabajo.