\resumenCastellano{
El objetivo de este trabajo de tesis es construir un \textit{framework} que permita la representación y construcción de Sistemas de Recomendación (SR). Estos sistemas producen recomendaciones personalizadas como salida, con el fin de guiar al usuario en la elección de productos o servicios útiles entre una gran cantidad de alternativas. Hoy en día abundan SR en las redes sociales en los dominios de las películas, libros, música entre otros y en el marcado \textit{(tagging)} de dicha información. Lamentablemente, los SR están cada vez más especializados para proveer recomendaciones de mayor calidad en su dominio de aplicación específico. Por otro lado, la Web 2.0 aporta nuevas formas de interacción \textit{(tagging, commenting, voting, rating)} a los SR. Todas estas nuevas fuentes de información mejoran las recomedaciones para los usuarios finales; pero, por otro lado pone en evidencia la falta de estándares para el desarrollo de este tipo de sistemas. Luego, ¿es posible contar con un \textit{framework} operacionalizable que permita hacer frente a la variedad de sistemas de recomendación existentes y mantenerlos todos bajo un esquema común de representación y construcción?. Para responder esta pregunta, esta tesis se apoya en un modelo de \textit{cooperative awareness} llamado \textit{3-ontology} proveniente del área de los sistemas colaborativos. El \textit{framework} permite modelar las características contextuales y la información social proveniente de la Web 2.0, con el objetivo de situar las interacciones en tres contenedores de sentido: comunidades, lugares y eventos. Para validar la eficacia del modelo propuesto se construye una API denomida RBOX 2.0 que permite representar y construir sistemas de recomendación. Los ejemplos de aplicación indican que el modelo propuesto permite representar y construir con eficacia sistemas de recomendación de filtrado colaborativo y \textit{tag clustering}. Como aporte al área de SR, el \textit{framework} y la API propuesta - construida en \textit{Java 7} - superan a las alternativas de representación y construcción de este tipo de sistemas que se encuentran en la literatura del área.

\vspace*{0.5cm}
\KeywordsES{Sistemas de Recomendación, API, Filtrado Colaborativo, 3-ontology, redes sociales}.
}

\newpage

\resumenIngles{
The objective of this thesis is to build a framework to allow the representation and construction of Recommender Systems. These systems produce as output personalized recommendations in order to guide the user in choosing useful products or services from a large number of alternatives. Today abound Recommendation Systems in social networks in the domain of movies, books, music and tagging of such information. Unfortunately, the SR are increasingly specialized to provide recommendations of quality in your specific application domain. On the other hand, Web 2.0 provides new forms of interaction tagging, commenting, voting, rating to SR. All these new sources of information improve recomendations for end users, but on the other hand highlights the lack of standards for the development of such systems . Then, is possible to have a operationalizable framework that can cope with the variety of existing recommendation systems and keep them all under a common representation scheme and construction? To answer this question, this thesis is based on a model of cooperative awareness called 3-Ontology coming from the Collaborative Systems area. The framework allows modeling contextual characteristics and social information from the Web 2.0, with the aim of identifying interactions in three sense containers: communities, places and events. To validate the effectiveness of the proposed model builds a API called RBOX 2.0 that allows to represent and build recommender systems. The case studies indicate that the proposed model can represent and effectively build recommender systems of collaborative filtering and tags clustering. As a contribution to the area of SR, the framework and the API -built in Java 7- proposal outweigh the alternative representation and construction of these systems in the literature.



\vspace*{0.5cm}
\KeywordsEN{Recommendation System, API, Collaborative Filtering, 3-ontology, social networks}.
}
