\resumenCastellano{
% Introducción
El procesamiento de transacciones de lecturas en motores de búsqueda demanda el uso eficiente de recursos de hardware para hacer frente a altas y dinámicas cargas de trabajo por parte de los usuarios. Estos sistemas son generalmente desplegados en grupos de procesadores. A medida que la Web crece, los motores de búsqueda toman mayor importancia en la búsqueda de información dentro de grandes cantidades de datos.

En el presente trabajo se abordan diferentes estrategias de procesamiento y de planificación de transacciones de lectura enfocados principalmente en (1) el acceso a grandes índices invertidos para obtener el conjunto de los mejores $K$ documentos para una consulta utilizando el algoritmo Wand, y (2) el uso de predictores de eficiencia para transacciones de lectura con el objetivo de reducir el tiempo de procesar lotes de consultas.  

Los resultados obtenidos muestran...

\vspace*{0.5cm}
\KeywordsES{recuperación de información, motores de búsqueda, Wand}.
}

\newpage

\resumenIngles{
Abstract—Processing queries in Web search engines demands the efficient use of hardware resources to cope with the scale and dynamics of user traffic. These systems are usually deployed on dedicated clusters of processors. As the Web becomes bigger, search engines are becoming increasingly important  to find information in large amounts of data. 

This work discusses different query processing and scheduling strategies, focused mainly on (1) access to large inverted index data structure to obtain the top $K$ most pertinent results for any query, using Wand algorithm, and (2) the use of different efficiency predictors for queries in order to process batches of queries.
 
The results show...



\vspace*{0.5cm}
\KeywordsEN{information retrieval, search engines, Wand}.
}
