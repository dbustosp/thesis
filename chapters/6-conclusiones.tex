\chapter{Conclusiones}
\label{cap:conclu}
En este capítulo se muestran las conclusiones del trabajo de tesis realizado. De esta forma, se explicitan los aportes realizados al área de los SR, se describe el cumplimiento de los objetivos específicos de este trabajo dando evidencia de su cumplimiento. Luego se presentan los posibles trabajos futuros a los que da lugar este trabajo de tesis. Finalmente, se exponen las reflexiones finales referentes al trabajo realizado.

\section{Aportes al \'area de los Sistemas de Recomendaci\'on}

El desarrollo de los SR como área de investigación ha sido acelerada en la última década debido al creciente desarrollo de la Web. En este trabajo se ha presentado un modelo de datos y un método de construcción de SR que otorga un marco de referencia para implementar SR. Dado esto, el primer aporte de este trabajo al área de los SR se basa en un modelo que permite una representación situada de los eventos que permite aprovechar las características contextuales de estos, este modelo tiene como base conceptual la \textit{3-Ontology} un \textit{framework} proveniente del área de los sistemas colaborativos. Luego se observa que los SR permiten capturar la inteligencia colectiva sobre un dominio de aplicación específico en base a tres contenedores de sentido comunidades, eventos y lugares. Estos además permiten persistir la inteligencia colectiva para ser usada posteriormente para entregar recomendaciones de utilidad a un usuario.

Un aporte de este trabajo de tesis corresponde a una herramienta de software llamada RBOX 2.0 que permite la construcción de SR basado en el modelo propuesto. RBOX 2.0 es de código abierto y extensible, permitiendo el desarrollo de SR tanto para el ámbito empresarial como científico. 

\section{Acerca de los objetivos espec\'ificos}

A continuación se describe el nivel de logro alcanzado en el desarrollo de este trabajo de tesis.
\begin{enumerate}
	\item \textit{Sistematizar el proceso de construcción de SR desde la literatura actual del área.}
	
	En la sección \ref{sec:consSR} se realizó una sistematización que permite la construcción de SR bajo el modelo propuesto. Este se basa en la evolución de los SR y sus representaciones presentadas en el marco teórico de este trabajo.
	
	\item \textit{Caracterizar las dimensiones emergentes para los SR.}
	
	En el capítulo 2 se presenta el marco teórico donde se definen las principales dimensiones emergentes de los SR. En el trabajo relacionado de la sección 4.1 se muestra como se han agregado nuevas dimensiones a los SR a las representaciones propuestas. 
	
	\item \textit{Diseñar un modelo orientado a eventos de representación de datos para SR.}
	
	En el capítulo 3 se presenta un modelo orientado a eventos que se basa en el \textit{framework 3-Ontology}. Este modelo permite situar los eventos dentro de un contexto social, espacial y temporal. En otras palabras este modelo es capaz de situar la colaboración existente dentro de aplicaciones pertenecientes a la Web 2.0.
	
	\item \textit{Diseñar un conjunto de operaciones para la obtención de datos desde el modelo de datos propuesto.}
	
	En el capítulo 3 se presenta la definición formal de un conjunto de operaciones que permiten obtener eventos, comunidades y lugares relevantes dependiendo del algoritmo de recomendación usado. Es importante notar que estas operaciones son parte del nivel meta de la \textit{3-Ontology} correspondientes a las trazas, mapas y retratos que dan sentido al nivel base (eventos, comunidades y lugares).
	
	\item \textit{Diseño e implementación de una herramienta para construir SR.}
	
	En el capítulo 4 se presenta el diseño e implementación de RBOX 2.0 una herramienta de software que permite el diseño e implementación de SR basados en el modelo de representación propuesto. RBOX 2.0 está construido bajo patrones de diseño que aseguran un conjunto de calidades sistemáticas requeridas. 
	
	\item \textit{Construir SR’s basados en el \textit{framework} propuesto.}
	
	En el capítulo 5 se presentan dos casos de estudio donde se presenta la construcción de dos SR basado en el modelo propuesto. El primer SR se basa en un algoritmo de vecinos más cercanos mediante el uso de eventos de tipo \textit{rating}, en este caso se recomiendan las películas con el \textit{rating} más alto predicho por el algoritmo. El segundo SR se basa en un algoritmo de \textit{clustering} basado en eventos de tipo \textit{tagging}, en este caso se recomiendan los \textit{tags} que se encuentren en el mismo \textit{cluster} que el \textit{tag} buscado. Dada la construcción de estos dos SR se valida la eficacia del modelo propuesto para la representación y construcción de SR basado en distintos tipos de eventos.
	
	\item \textit{Publicar los resultados de la investigación en una revista de la especialidad.}
	
	Se encuentra en desarrollo 2 publicaciones. La primera se basa en el modelo de representación de datos propuesto. La segunda es el diseño e implementación de RBOX 2.0 como herramienta para la representación y construcción de SR.
	
	
\end{enumerate}

\section{Trabajo futuro}

En este trabajo se propone e implementa un modelo de representación y construcción de SR basado en el \textit{framework} conceptual \textit{3-Ontology}. En este trabajo se exploró la eficacia del \textit{framework} para construir SR sin utilizar todas sus dimensiones, dado esto se vislumbran los siguientes trabajos futuros de índole científico e ingenieril:

\begin{itemize}
\item Investigar la dimensión de la comunidad y como esta permite retro-alimentar el proceso de recomendación. Por ejemplo,detectar comunidades emergentes dadas las valoraciones de los usuarios y almacenarlas dentro del modelo, para luego ser usadas como información adicional para futuras recomendaciones. Este trabajo se encuentra bajo desarrollo como tesis de Magíster en Ingeniería Informática \citep{Ochoa:2013}.

\item Investigar la dimensión referente a los lugares mediante la propuesta de un \textit{framework} que explicite los tipos de información espacial que son usados en SR. Construir un conjunto de algoritmos basados en lugares y explicitar las mejoras obtenidas en los SR.

\item Realizar una expansión del \textit{framework} propuesto para la evaluación de SR de forma \textit{offline} y \textit{online}, esto otorgaría una forma unificada para comparar diversos SR.

\item Investigar la construcción de SR híbridos utilizando las tres dimensiones del \textit{framework} para validar la eficacia y eficiencia de construir SR de recomendación con el modelo propuesto.

\item Estudiar la capacidad  de la \textit{3-Ontology} para representar la Web 2.0. Esto se justifica en la capacidad de representación de entornos colaborativos que posee la \textit{3-Ontology}. En este trabajo solo estudio su utilización para SR, sin embargo se podría extender en un meta-modelo para todo tipo de aplicación de la Web 2.0.

\item La construcción de un IDE basado en el modelo propuesto utilizando la implementación de RBOX 2.0. Esto facilitaría la construcción dado el uso de una interfaz gráfica que guié el proceso de construcción.

\end{itemize}

\section{Reflexiones finales}

El desarrollo de este trabajo de tesis ha significado un gran desafio personal. Desde un comienzo he participado del grupo \textit{Social Tagging} perteneciente al proyecto FONDEF D09I1185 Observatorios de la Web en tiempo real, en este equipo incentivados por el Dr. Edmundo Leiva adoptamos un trabajo colaborativo que nos permitió compartir nuestro trabajo entregando y recibiendo ayuda de los miembros del equipo. Esto ha sido una experiencia enriquecedora donde se vive problemas similares a los que tendríamos en nuestra vida profesional. 

En este trabajo he tenido que utilizar las capacidades ingenieriles y científicas adquiridas durante mis años de formación en el departamento de informática. Se valora de manera muy especial la entrega de habilidades blandas que permiten un mejor desempeño en el ámbito social y laboral.

Sobre el tema abordado en este trabajo ha sido muy interesante debido a su impacto en la sociedad actual, donde la Web es parte de nuestro diario vivir y los SR nos permiten tener a nuestra disposición información relevante. Dado lo anterior para ámbitos empresariales es necesario contar con representaciones y estándares que permitan construir SR.
